\documentclass[a4paper,12pt]{article}

\usepackage[utf8]{inputenc}
\usepackage[T1,T2A]{fontenc}  % Font encoding including Cyrillic
\usepackage[serbianc]{babel}  % Serbian language

\usepackage{geometry}
\usepackage{graphicx}
\usepackage{amsmath}
\usepackage{float}
\usepackage{hyperref}
\usepackage{listings}

% Adjust page margins if needed
\geometry{top=1in, bottom=1in, left=1in, right=1in}

\title{Једначина провођења топлоте}
\author{Бошко Андрић 26/2020 \\ Лука Филиповић 83/2020}
\date{\today}

\begin{document}
	
	\maketitle
	\tableofcontents
	
	\pagebreak
	
	\section{Увод}
	
	Провођење топлоте, основни феномен у области физике, често се изучава кроз математичко моделирање користећи парцијалне диференцијалне једначине. Овај пројекат има за циљ истраживање и решавање једначине провођења топлине у једној и две димензије примењујући нумеричку методу коначних разлика.
	\\
	\\
	Наша пажња усмерена је на динамику температуре унутар штапа или шипке, односно плоче, са почетним и граничним условима који често карактеришу реалне сценарије. Овај приступ нам омогућава дубље разумевање утицаја различитих параметара, као што је термална дифузивност, на промене температуре током времена и простора.
	
	
	\section{Поставка проблема}
	
	\begin{center}
		\textbf{Поставка Проблема - Једна Димензија}
	\end{center}
	
	У овом пројекту истраживања провођења топлоте у једној димензији, анализираћемо динамику температуре у штапу дужине \textit{L}. Примењиваћемо једначину провођења топлоте изражену као:\\\begin{center}
		$ \frac{\partial T}{\partial t} = \alpha \frac{\partial^2 T}{\partial x^2} $
	\end{center} ,где је \textit{T} температура, \textit{t} време, \textit{х} просотрна координата, а \textit{$\alpha$} термална дифузивност.\\ Задаћемо почетни услов $ T(x,0) = f(x) $, као и граничне услове $ T(0,t) = {T}_{0} $ и $ T(L,t) = {T}_{L} $. Циљ је пратити како се температура мења унутар штапа током времена.
	
	\begin{center}
		\textbf{Поставка Проблема - Две Димензије}
	\end{center}
	У случају две димензије, истраживање провођења топлоте обавићемо у блоку димензија \textit{${L}_{x}$} по \textit{x} координати и \textit{${L}_{y}$} по \textit{y} координати. Једначина провођења топлоте сада ће бити изражена као:\\ \begin{center}
		$ \frac{\partial T}{\partial t} = \alpha (\frac{\partial^2 T}{\partial x^2} + \frac{\partial^2 T}{\partial y^2}) $
	\end{center} Почетни услов ће бити $ T(x,y,0) = f(x,y) $, а гранични услови могу бити задати за оба просторна правца \textit{x} и \textit{y}. На пример, $ T(0,y,t) = {T}_{0} $, $ T({L}_{x},y,t) = {T}_{L}$, $ T(x,0,t) = {T}_{0} $, $ T(x,{L}_{y},t) = {T}_{L}$. Исто тако, циљ је пратити промене температуре унутар блока током времена.    
	
	\section{Метода коначних разлика}
	
	Метода коначних разлика је нумеричка техника која се користи за решавање парцијалних диференцијалних једначина. У нашем пројекту, применили смо методу коначних разлика на једначину провођења топлоте како бисмо добили апроксимацију решења.
	
	\subsection{Дискретизација топлотне једначине}
	
	Први корак у примени методе коначних разлика је дискретизација топлотне једначине. Ми смо дискретизовали простор и време, користећи одговарајуће кораке (\( \Delta x \) и \( \Delta t \)). 
	
	За једначину провођења топлоте у једној димензији, дискретизовали смо други производ по простору и први производ по времену помоћу централних разлика:
	
	\[ \frac{T_{i}^{n+1} - T_{i}^{n}}{\Delta t} = \alpha \frac{T_{i+1}^{n} - 2T_{i}^{n} + T_{i-1}^{n}}{(\Delta x)^2} \]
	
	Овде, индекс \( i \) представља просторну координату, а индекс \( n \) временску координату.
	\\\\
	Аналогно и за једначину провођења топлоте у две димензије:
	
	\[ \frac{T_{i,j}^{n+1} - T_{i,j}^{n}}{\Delta t} = \alpha \left( \frac{T_{i+1,j}^{n} - 2T_{i,j}^{n} + T_{i-1,j}^{n}}{(\Delta x)^2} + \frac{T_{i,j+1}^{n} - 2T_{i,j}^{n} + T_{i,j-1}^{n}}{(\Delta y)^2} \right) \]
	
	
	\subsection{Гранични и почетни услови}
	
	Након дискретизације, морамо узети у обзир и граничне услове. За граничне услове у једној димензији, користили смо претходно задате температуре на крајевима штапа (\(T_0\) и \(T_L\)):
	
	\[ T_0 = T_0 \]
	\[ T_L = T_L \]
	
	Што се тиче почетних услова, користили смо задату почетну температуру \( f(x) \).\\
	
	
	У случају једначине провођења топлоте у две димензије, корисимо следеће граничне услове:
	
	\[ T_{0,j} = T_0 \]
	\[ T_{L_x,j} = T_{L_x} \]
	\[ T_{i,0} = T_1 \]
	\[ T_{i,L_y} = T_{L_y} \]
	
	\subsection{Итеративно Решавање}
	
	Након дискретизације и узимања у обзир граничних и почетних услова, решавање једначине провођења топлоте своди се на итеративни процес. Користили смо одговарајуће алгоритме за ажурирање температуре у сваком кораку времена, усмеравајући се ка конвергенцији ка стабилном решењу.
	
	
\end{document}
